\chapter{System Requirements}
\label{sec:results}
\thispagestyle{fancy}
The project assignment stated by the TDT4215 cours staff was to create a search application, implemented either in Python or in Java, consisting of a basic system and an extended system. Only approved libraries could be used in the final application, other libraries not listed on the course web page could be approved by contacting the staff members. Some of the specific challenges were phrase search and proximity.

The project requirements stated by stuff were as follows:
\begin{itemize}
	\item The preprocessing should include tokenization, stopword removal and stemming.
	\item The indexing and query retrieval should use the cosine vector model and inverted files that must be stored and loaded on startup.
	\item The resulting application should use a clearly defined query language.
	\item The query result should be sorted and presented to the user according to the similarity ranking, the result should also include a link to the source document.
	\item The final implementation should be evaluated on time collection with 10 defined queries.
	\item The extended system should use a clustering technique to improve the search quality, the document clusters must also be ranked according to a similarity measure.
	\item The project, report and presentation deadlines were set to 39 days (5 weeks) from the project start.
	\item The number of group members were limited to maximum five persons.
\end{itemize}